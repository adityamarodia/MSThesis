\section{Schemes and Sheaves}
\section{Morphisms}
\section{Reduced and Irreducible Schemes (Varieties)}
\section{Fibred Products}
\section{Categories - Yoneda and Representable Functors}
\section{Some Moduli Theory}
The goal of a moduli problem is to classify a collection of geometric objects upto some sort of equivalence.
The \textit{geometric object} can vary significantly from one moduli problem to another. 
For example, in the rest of this thesis we will majorly be dealing with two major moduli problems - one where the object in consideration is a rational plane curve, and the other where the rational plane curve comes together with its embedding $\mu$ in $\mathbb{P}^{r}$.
\par Next, we need the notion of a \textit{family} of these geometric objects. A family of geometric objects is a collection of \textit{similar} geometric objects. For example, the set of (non vertical) lines in $\mathbb{R}^{2}$:
\begin{align*}
    \mathcal{X} = \{\{(x, ax + b): x \in \mathbb{R}\} : (a,b) \in \mathbb{R}^{2}\}
\end{align*}
Each object in this set $\mathcal{X}$ is a line given by the equation $y = ax + b$.
Notice that this \textit{family} $\mathcal{X}$ is parameterised by points in $\mathbb{R}^{2}$, that is, we have a map $\varphi: \mathcal{X} \to \mathbb{R}^{2}$, which sends the line $y = ax + b$ to $(a,b)$.  
Therefore, we can think of the family of nonvertical lines in $\mathbb{R}^{2}$ as the map:
\[\begin{tikzcd}
	{\mathcal{X}} \\
	\\
	{\mathbb{R}^2}
	\arrow["\varphi",from=1-1, to=3-1]
\end{tikzcd}\]
With the fibre $\varphi^{-1}(p)$ over the point $p = (a,b)$ being the line $y = ax + b$.
An advantage of describing the set of lines $\mathcal{X}$ with this parameterisation is that points close to each other in $\mathbb{R}^{2}$ correspond to lines which are "close to each other".  
Hence adding additional structure to our \textit{family}. 
This motivates our definiton for a family.
A morphism $\mathcal{X} \to B$ betwenn schemes (with possibly additional structure) will be referred to as a family over the scheme $B$.

Finally, the notion of equivalence we need is an equivalence relation between the set of families over a base space $B$.
This finishes the idea behind a moduli problem.


\section{Blowups}
\section{Flatness?}
