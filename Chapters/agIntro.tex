%\section{Schemes and Sheaves}
%\section{Morphisms}
%\section{Reduced and Irreducible Schemes (Varieties)}
%\section{Fibred Products}
\section{Categories - Yoneda and Representable Functors}
\section{Some Moduli Theory}
The goal of a moduli problem is to classify a collection of geometric objects upto some sort of equivalence.
The \textit{geometric object} can vary significantly from one moduli problem to another. 
For example, in the rest of this thesis we will majorly be dealing with two major moduli problems - one where the object in consideration is a rational plane curve, and the other where the rational plane curve comes together with its embedding $\mu$ in $\mathbb{P}^{r}$.

Before I what exactly a Moduli problem entails, consider the following example:

\begin{example}
    \label{exLinesInPlane}
\begin{align*}
    \mathcal{X} = \{\{(x, ax + b): x \in \mathbb{R}\} : (a,b) \in \mathbb{R}^{2}\}
\end{align*}
Each object in this set $\mathcal{X}$ is a line given by the equation $y = ax + b$.
Notice that this set $\mathcal{X}$ is parameterised by points in $\mathbb{R}^{2}$, that is, we have a map $\varphi: \mathcal{X} \to \mathbb{R}^{2}$, which sends the line $y = ax + b$ to $(a,b)$.  
Therefore, we can think of the family of nonvertical lines in $\mathbb{R}^{2}$ as the map:
\[\begin{tikzcd}
	{\mathcal{X}} \\
	\\
	{\mathbb{R}^2}
	\arrow["\varphi",from=1-1, to=3-1]
\end{tikzcd}\]
With the fibre $\varphi^{-1}(p)$ over the point $p = (a,b)$ being the line $y = ax + b$.
An advantage of describing the set of lines $\mathcal{X}$ with this parameterisation is that points close to each other in $\mathbb{R}^{2}$ correspond to lines which are ``close to each other".  

\end{example}

This example motivates the definition of a family: 
\begin{definition}
    A morphism $\mathcal{X} \to B$ between schemes (with possibly additional structure) will be referred to as a family over the scheme $B$. This will often be denoted as $\mathcal{X}/B$.
\end{definition}
The fibres over points of the scheme $B$ correspond to out geometric objects, in context of example \ref{exLinesInPlane} the fibres were non-vertical lines in $\mathbb{R}^{2}$.
\par It's usually easier to deal with families when we define a notion of equivalent families. 
That is, an equivalence relation on the set of families, $S(B)$, over the scheme $B$. 
Since we are dealing with schemes, an obvious notion of equivalence between families $\mathcal{X}/B$ and $\mathcal{X}'/B$ can be defined with isomorphisms. That is, an isomorphism, $\psi: \mathcal{X} \to \mathcal{X}'$ such that the following diagram commutes:
\[
\begin{tikzcd}
    {\mathcal{X}} && {\mathcal{X}'}\\
    \\
                  & {B} 
     \arrow["\psi", from = 1-1, to = 1-3]
     \arrow[from = 1-1, to = 3-2]
     \arrow[from = 1-3, to = 3-2]
\end{tikzcd}
\]
Note that isomorphisms over the scheme $B$ is just an example of an equivalence relation on $S(B)$, over the rest of the thesis we will encounter different equivalence realtions which are appropriate for the given situations.
When two families $\mathcal{X}/B$ and $\mathcal{N}/B$ lie in the same equivalence class in $S(B)$, we write $\mathcal{X}/B \simeq \mathcal{N}/B$.
\par Finally, a moduli problem consists of:
\begin{itemize}
    \item A class of geometric objects.
    \item A family of these geometric objects over a parameter space (base scheme).
    \item And an equivalence relation on the set of families, $S(B)$, over a base $B$.
\end{itemize}

An important idea while dealing with families of objects is being able to change your base space.
If we have a family $\mathcal{X}/B$ and a morphism $\varphi: B' \to B$ then one can define a new family over the base $B'$ given by the fibre product $\mathcal{X} \times_{B} B' \to B'$.
\[
\begin{tikzcd}
    {\varphi^{*}\mathcal{X} := \mathcal{X} \times_{B} B'} & & \mathcal{X}\\
    \\
    B' && B
    \arrow[ from = 1-3, to = 3-3]
    \arrow[from = 1-1, to = 1-3]
    \arrow["\varphi", from = 3-1, to = 3-3]
    \arrow[from = 1-1, to = 3-1]
\end{tikzcd}
\]
The family $\varphi^{*}\mathcal{X}/B'$ is called the pullback family of $\mathcal{X}/B$ along the morphism $\varphi$.
For our families to behave ``nicely" we would like pullbacks to commute with equivalence. 
That is, if $\mathcal{X}/B \simeq \mathcal{N}/B $ and $\varphi B' \to B$ is a morphism then we would like:
\begin{align*}
    \varphi^{*}\mathcal{X}/B' \simeq \varphi^{*}\mathcal{N}/B'.
\end{align*}

Before we formally define a moduli problem in the context of categories let's look at example \ref{exLinesInPlane} carefully.
\begin{example}
    \label{compExLinesInplane}
    In example \ref{exLinesInPlane} we defined a parameter space for non-vertical lines in $\mathbb{R}^{2}$. 
    But we would like to define a parameter/base space such that we can characterize all lines (including vertical ones) in $\mathbb{R}^{2}$.
    \par In the diagram above we have embedded $\mathbb{R}^{2}$ in $\mathbb{R}^{3}$ as the plane $z = 1$.
    For each line $ax + by + c = 0 $, there exists a plane, $L := ax + by + cz = 0$,  through the origin in $\mathbb{R}^{3}$ such that $L \cap \{z=1\}$ corresponds to the line $ax + by + c = 0 $.
    \par We also know that each plane through the origin, $\mathbb{R}^{3}$, corresponds to a line through the origin (its normal vector). 
    This gives a bijection between the set of lines in $\mathbb{R}^{2}$ and the points of $\mathbb{RP}^{2}\backslash \{ [0:0:1]\}$, which can be seen as the family:
    \[
        \begin{tikzcd}
            \mathcal{Y} := \{\text{set of lines in } \mathbb{R}^{2}\}\\
            \\
            \mathbb{RP}^{2}\backslash {[0:0:1]}
            \arrow["\psi", from = 1-1, to = 3-1]
        \end{tikzcd}
    \]
    Notice that we are one point away from our parameter space being the entire projective space $\mathbb{RP}^{2}$.
    It's natural to ask what space does $\mathbb{RP}^{2}$ parameterise?
    Luckily the answer to this question is quite striaghtforward.
    We have a bijection between lines in $\mathbb{RP}^{2}$ and points in $\mathbb{RP}^{2}$ given by:
    \[
        \begin{tikzcd}
            {\mathbb{RP}^{2}} & & {\{\text{lines in }\mathbb{RP}^{2}\}} \\
        {[a:b:c]} \arrow[u, phantom, sloped, "\in"]& & {\{[x:y:z] \in \mathbb{RP}^{2}: ax + by + cz = 0 \}}\arrow[u, phantom, sloped, "\in"].
            \arrow["\gamma", leftarrow, from = 1-1, to = 1-3]
            \arrow[ leftrightarrow, from = 2-1, to = 2-3]
            %\arrow[phantom, from = 2-1, to = 1-1, sloped, "\in"]
            %\arrow[phantom, from = 2-2, to = 1-2, sloped, "\in"]
        \end{tikzcd}
    \]
    And this can be seen as a family $\gamma : \mathbb{RP}^{2}\to \mathbb{RP}^{2}$ of lines in $\mathbb{RP}^{2}$.
    \par We can now interpret the families $\mathcal{X}/\mathbb{R}^{2}$ (from example \ref{exLinesInPlane}) and $\mathcal{Y}/\{\mathbb{RP}^{2}\backslash {[0:0:1]}\} $ as families of lines in $\mathbb{RP}^{2}$ by embedding $\mathbb{R}^{2}$ in $\mathbb{RP}^{2}$, giving us the three families:
    \begin{itemize}
        \item $\varphi : \mathcal{X} \to \mathbb{R}^{2}$
        \item $\psi: \mathcal{Y} \to \mathbb{RP}^{2}\backslash {[0:0:1]}\} $
        \item $\gamma: \mathbb{RP}^{2} \to \mathbb{RP}^{2}$.
    \end{itemize}
    It turns out that under the inclusion maps, 
    \begin{itemize}
        \item $i: \mathbb{R}^{2}\hookrightarrow \mathbb{RP}^{2}$,
        \item $j: \mathbb{RP}^{2}\backslash {[0:0:1]}\} \hookrightarrow \mathbb{RP}^{2}$,
    \end{itemize}
    we have: 
    \begin{align*}
        i^{*}\mathbb{RP}^{2} \simeq &\mathcal{X}\\
        j^{*}\mathbb{RP}^{2} \simeq &\mathcal{Y}.
    \end{align*}
    It turns out that for any family of lines in $\mathbb{RP}^{2}$, $\mathcal{Z}/B$, there exists a unique morphism $\phi :B \to \mathbb{RP}^{2}$, such that $\mathcal{Z}/B$ is equivalent to the pullback family along $\phi$.
    In the next section we will see that the family $\gamma: \mathbb{RP}^{2} \to \mathbb{RP}^{2}$ satisfies the properties of a universal family, with the base space being defined as a \textbf{fine moduli space}.

\end{example}


\section{Blowups}
\section{Flatness?}
