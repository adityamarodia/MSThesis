%\section{Schemes and Sheaves}
%\section{Morphisms}
%\section{Reduced and Irreducible Schemes (Varieties)}
%\section{Fibred Products}
\section{Moduli Theory}
\subsection{Introduction and Motivation}
The goal of a moduli problem is to classify a collection of geometric objects upto some sort of equivalence.
The \textit{geometric object} can vary significantly from one moduli problem to another. 
For example, in the rest of this thesis we will majorly be dealing with two moduli problems - one where the object in consideration is a rational plane curve, and the other where the rational plane curve comes together with its embedding $\mu$ in $\mathbb{P}^{r}$.

Before I what exactly a Moduli problem entails, consider the following example:

\begin{example}
    \label{exLinesInPlane}
\begin{align*}
    \mathcal{X} = \{\{(x, ax + b) \in \mathbb{R}^{2}: x \in \mathbb{R}\} : (a,b) \in \mathbb{R}^{2}\}
\end{align*}
Each object in this set $\mathcal{X}$ is a line given by the equation $y = ax + b$.
Notice that this set $\mathcal{X}$ is parameterised by points in $\mathbb{R}^{2}$, that is, we have a map $\varphi: \mathcal{X} \to \mathbb{R}^{2}$, which sends the line $y = ax + b$ to $(a,b)$.  
Therefore, we can think of the family of nonvertical lines in $\mathbb{R}^{2}$ as the map:
\[\begin{tikzcd}
	{\mathcal{X}} \\
	\\
	{\mathbb{R}^2}
	\arrow["\varphi",from=1-1, to=3-1]
\end{tikzcd}\]
With the fibre $\varphi^{-1}(p)$ over the point $p = (a,b)$ being the line $y = ax + b$.
An advantage of describing the set of lines $\mathcal{X}$ with this parameterisation is that points close to each other in $\mathbb{R}^{2}$ correspond to lines which are ``close to each other".  

\end{example}

This example motivates the definition of a family: 
\begin{definition}
    A morphism $\mathcal{X} \to B$ between schemes (with possibly additional structure) will be referred to as a family over the scheme $B$. This will often be denoted as $\mathcal{X}/B$.
\end{definition}
The fibres over points of the scheme $B$ correspond to out geometric objects, in context of example \ref{exLinesInPlane} the fibres were non-vertical lines in $\mathbb{R}^{2}$.
\par It's usually easier to deal with families when we define a notion of equivalent families. 
That is, an equivalence relation on the set of families, $S(B)$, over the scheme $B$. 
Since we are dealing with schemes, an obvious notion of equivalence between families $\mathcal{X}/B$ and $\mathcal{X}'/B$ can be defined with isomorphisms. That is, an isomorphism, $\psi: \mathcal{X} \to \mathcal{X}'$ such that the following diagram commutes:
\[
\begin{tikzcd}
    {\mathcal{X}} && {\mathcal{X}'}\\
    \\
                  & {B} 
     \arrow["\psi", from = 1-1, to = 1-3]
     \arrow[from = 1-1, to = 3-2]
     \arrow[from = 1-3, to = 3-2]
\end{tikzcd}
\]
Note that isomorphisms over the scheme $B$ is just an example of an equivalence relation on $S(B)$, over the rest of the thesis we will encounter different equivalence realtions which are appropriate for the given situations.
When two families $\mathcal{X}/B$ and $\mathcal{N}/B$ lie in the same equivalence class in $S(B)$, we write $\mathcal{X}/B \simeq \mathcal{N}/B$.
\par Finally, a moduli problem consists of:
\begin{itemize}
    \item A class of geometric objects.
    \item A family of these geometric objects over a parameter space (base scheme).
    \item And an equivalence relation on the set of families, $S(B)$, over a base $B$.
\end{itemize}

An important idea while dealing with families of objects is being able to change your base space.
If we have a family $\mathcal{X}/B$ and a morphism $\varphi: B' \to B$ then one can define a new family over the base $B'$ given by the fibre product $\mathcal{X} \times_{B} B' \to B'$.
\[
\begin{tikzcd}
    {\varphi^{*}\mathcal{X} := \mathcal{X} \times_{B} B'} & & \mathcal{X}\\
    \\
    B' && B
    \arrow[ from = 1-3, to = 3-3]
    \arrow[from = 1-1, to = 1-3]
    \arrow["\varphi", from = 3-1, to = 3-3]
    \arrow[from = 1-1, to = 3-1]
\end{tikzcd}
\]
The family $\varphi^{*}\mathcal{X}/B'$ is called the pullback family of $\mathcal{X}/B$ along the morphism $\varphi$.
For our families to behave ``nicely" we would like pullbacks to commute with equivalence. 
That is, if $\mathcal{X}/B \simeq \mathcal{N}/B $ and $\varphi B' \to B$ is a morphism then we would like:
\begin{align*}
    \varphi^{*}\mathcal{X}/B' \simeq \varphi^{*}\mathcal{N}/B'.
\end{align*}

Before we formally define a moduli problem in the context of categories let's look at example \ref{exLinesInPlane} carefully.
\begin{example}
    \label{compExLinesInplane}
    In example \ref{exLinesInPlane} we defined a parameter space for non-vertical lines in $\mathbb{R}^{2}$. 
    But we would like to define a parameter/base space such that we can characterize all lines (including vertical ones) in $\mathbb{R}^{2}$.
    \par In the diagram above we have embedded $\mathbb{R}^{2}$ in $\mathbb{R}^{3}$ as the plane $z = 1$.
    For each line $ax + by + c = 0 $, there exists a plane, $L := ax + by + cz = 0$,  through the origin in $\mathbb{R}^{3}$ such that $L \cap \{z=1\}$ corresponds to the line $ax + by + c = 0 $.
    \par We also know that each plane through the origin, $\mathbb{R}^{3}$, corresponds to a line through the origin (its normal vector). 
    This gives a bijection between the set of lines in $\mathbb{R}^{2}$ and the points of $\mathbb{RP}^{2}\backslash \{ [0:0:1]\}$, which can be seen as the family:
    \[
        \begin{tikzcd}
            \mathcal{Y} := \{\text{set of lines in } \mathbb{R}^{2}\}\\
            \\
            \mathbb{RP}^{2}\backslash {[0:0:1]}
            \arrow["\psi", from = 1-1, to = 3-1]
        \end{tikzcd}
    \]
    Notice that we are one point away from our parameter space being the entire projective space $\mathbb{RP}^{2}$.
    It's natural to ask what space does $\mathbb{RP}^{2}$ parameterise?
    Luckily the answer to this question is quite striaghtforward.
    We have a bijection between lines in $\mathbb{RP}^{2}$ and points in $\mathbb{RP}^{2}$ given by:
    \[
        \begin{tikzcd}
            {\mathbb{RP}^{2}} & & {\{\text{lines in }\mathbb{RP}^{2}\}} \\
        {[a:b:c]} \arrow[u, phantom, sloped, "\in"]& & {\{[x:y:z] \in \mathbb{RP}^{2}: ax + by + cz = 0 \}}\arrow[u, phantom, sloped, "\in"].
            \arrow["\gamma", leftarrow, from = 1-1, to = 1-3]
            \arrow[ leftrightarrow, from = 2-1, to = 2-3]
            %\arrow[phantom, from = 2-1, to = 1-1, sloped, "\in"]
            %\arrow[phantom, from = 2-2, to = 1-2, sloped, "\in"]
        \end{tikzcd}
    \]
    And this can be seen as a family $\gamma : \mathbb{RP}^{2}\to \mathbb{RP}^{2}$ of lines in $\mathbb{RP}^{2}$.
    \par We can now interpret the families $\mathcal{X}/\mathbb{R}^{2}$ (example \ref{exLinesInPlane}) and $\mathcal{Y}/\{\mathbb{RP}^{2}\backslash {[0:0:1]}\} $ as families of lines in $\mathbb{RP}^{2}$ by embedding $\mathbb{R}^{2}$ in $\mathbb{RP}^{2}$, giving us the three families:
    \begin{itemize}
        \item $\varphi : \mathcal{X} \to \mathbb{R}^{2}$
        \item $\psi: \mathcal{Y} \to \mathbb{RP}^{2}\backslash {[0:0:1]}\} $
        \item $\gamma: \mathbb{RP}^{2} \to \mathbb{RP}^{2}$.
    \end{itemize}
    For the inclusion maps, 
    \begin{itemize}
        \item $i: \mathbb{R}^{2}\hookrightarrow \mathbb{RP}^{2}$,
        \item $j: \mathbb{RP}^{2}\backslash {[0:0:1]}\} \hookrightarrow \mathbb{RP}^{2}$,
    \end{itemize}
    we have: 
    \begin{align*}
        i^{*}\mathbb{RP}^{2} \simeq &\mathcal{X}\\
        j^{*}\mathbb{RP}^{2} \simeq &\mathcal{Y}.
    \end{align*}
    Further, for any family of lines in $\mathbb{RP}^{2}$, $\mathcal{Z}/B$, there exists a unique morphism $\phi :B \to \mathbb{RP}^{2}$, such that $\mathcal{Z}/B$ is equivalent to the pullback family along $\phi$.
    We will see that the family $\gamma: \mathbb{RP}^{2} \to \mathbb{RP}^{2}$ satisfies the properties of a \textbf{universal family}, with the base space being defined as a \textbf{fine moduli space}.
\end{example}

\begin{definition}[Fine moduli spaces]
    A \textbf{universal family} for a moduli problem is a family $U/M$ such that for any other family $\mathcal{X}/B$ there exists a unique morphism $\varphi: B \to M$ with the property that $\varphi^{*}U \simeq \mathcal{X}$.
    \par The base space of the universal family $M$ is called a \textbf{fine moduli space}. The morphism $\varphi$ is called the \textbf{classifying map} of the family $\mathcal{X}/B$.
\end{definition}

\subsection{Moduli Problems in the language of Categories}
The properties of a moduli problem, like commutativity of pullback families with equivalences, can be stated elegantly in the language of categories.
To see this, consider the following commutative diagram:
\[
\begin{tikzcd}
    {\varphi^{*}\mathcal{X} := \mathcal{X} \times_{B} B'} & & \mathcal{X}\\
    \\
    B' && B
    \arrow[ from = 1-3, to = 3-3]
    \arrow[from = 1-1, to = 1-3]
    \arrow["\varphi", from = 3-1, to = 3-3]
    \arrow[from = 1-1, to = 3-1]
\end{tikzcd}
\]
Given a morphism $\varphi: B' \to B$ we get a map from $S(B)$ to $S(B')$ which sends $\mathcal{X}/B$ to $\varphi^{*}\mathcal{X}/B'$. 
Remember that $S(B)$ denotes the equivalence class of families over $B$ - hence ensuring that pullback is defined upto equivalence.
We can interpret this reversing of directions as a contravariant functor which sends each scheme to the equivalence class of families over it. 
Consider a functor $S$ defined as:
    \begin{align*}
        S:\textbf{Sch}^{op} & \to \textbf{Sets}\\
        X &\mapsto S(X):=\{\text{Equivalence classes of families over }X\},
    \end{align*}
where for $\varphi \in \text{Mor}(A,B)$, $S(\varphi) = \varphi^{*} \in \text{Mor}(S(B),S(A))$. 
\subsubsection{Representable Funtors and Fine Moduli Spaces}
\begin{definition}[Functor of Points]
    For any scheme $X$ there exists a contravariant set-valued functor, $h_{X}$, called its \textbf{funtor of points}.
    This funtor is defined as: 
    \begin{align*}
        h_{X}: \textbf{Sch}^{op} &\to \textbf{Sets}\\
         B &\mapsto \emph{Hom}(B,X),
    \end{align*}
where for $\varphi:B' \to B$, $h_{X}(\varphi)$ is defined as:
\begin{align*}
    h_Y(\varphi):\emph{Hom}(B,X) &\to \emph{Hom}(B',Y)\\
    \beta &\mapsto \beta \circ \varphi
\end{align*}
\end{definition}
\begin{definition}[Representable Functors]
    A functor $S$ isomorphic to a functor points (for some scheme $X$) is called a \textbf{representable functor}.
    If $\mathcal{U}:h_{Y}\to S$ is an isomrphism of functors, then we say that the pair $(Y, \mathcal{U})$ represents $S$.
\end{definition}
By Yoneda's Lemma (appendix) we knoww that we can identify the natiral transformation $\mathcal{U}:h_{Y}\to S$ with an element of $S(Y)$.
Suppose the element of $S(Y)$ corresponding to $\mathcal{U}$ is $U$.
Then $(Y,U)$ is said to represent $S$.
\begin{proposition}
    A family $U/M$ is the universal family (with M the fine moduli space) for a moduli problem, if and only if the pair $(M,U)$ represents the moduli functor $S$.
\end{proposition}
\begin{proof}
    \textcolor{red}{add the proof}
\end{proof}

Notice that the previous proposition doesn't deal with the existence of fine moduli spaces in general.
There are many interesting moduli functors (one which we will encounter later) in the thesis which aren't representable.
That is, fine moduli spaces need not exist for such moduli problems.
It's now natural to ask if there is a way we can relax the definition of a fine moduli space to construct some sort of a \textit{best approximation}? We will call this approximation the \textbf{coarse moduli space}.

\begin{definition}
    A \textbf{coarse moduli space} for a moduli functor $S$ is the pair $(M, \mathcal{V})$, where $M$ is a scheme and $\mathcal{V}:S \to h_{M}$ is a natural transform of functor such that:
    \begin{itemize}
        \item $(M,\mathcal{V})$ is initial across all such pairs,
        \item The map between sets, 
            \begin{align*}
                \mathcal{V}_{\emph{Spec}\mathbb{C}}: S(\emph{Spec}\mathbb{C}) \to \emph{Hom}(\emph{Spec}\mathbb{C}, M),
            \end{align*}
            is a bijection.
    \end{itemize}
\end{definition}

\section{Blowups}
\section{Flatness?}
