\section{Moduli Theory}
\subsection{Introduction and Motivation}
The goal of a moduli problem is to classify a collection of geometric objects upto some sort of equivalence.
The \textit{geometric object} can vary significantly from one moduli problem to another. 
For example, in the rest of this thesis we will majorly be dealing with two moduli problems - one where the object in consideration is a rational plane curve, and the other where the rational plane curve comes together with its embedding $\mu$ in $\mathbb{P}^{r}$.

Before I what exactly a Moduli problem entails, consider the following example:

\begin{example}
    \label{exLinesInPlane}
\begin{align*}
    \mathcal{X} = \{\{(x, ax + b) \in \mathbb{R}^{2}: x \in \mathbb{R}\} : (a,b) \in \mathbb{R}^{2}\}
\end{align*}
Each object in this set $\mathcal{X}$ is a line given by the equation $y = ax + b$.
Notice that this set $\mathcal{X}$ is parameterised by points in $\mathbb{R}^{2}$, that is, we have a map $\varphi: \mathcal{X} \to \mathbb{R}^{2}$, which sends the line $y = ax + b$ to $(a,b)$.  
Therefore, we can think of the family of nonvertical lines in $\mathbb{R}^{2}$ as the map:
\[\begin{tikzcd}
	{\mathcal{X}} \\
	\\
	{\mathbb{R}^2}
	\arrow["\varphi",from=1-1, to=3-1]
\end{tikzcd}\]
With the fibre $\varphi^{-1}(p)$ over the point $p = (a,b)$ being the line $y = ax + b$.
An advantage of describing the set of lines $\mathcal{X}$ with this parameterisation is that points close to each other in $\mathbb{R}^{2}$ correspond to lines which are ``close to each other".  

\end{example}

This example motivates the definition of a family: 
\begin{definition}
    A morphism $\mathcal{X} \to B$ between schemes (with possibly additional structure) will be referred to as a family over the scheme $B$. This will often be denoted as $\mathcal{X}/B$.
\end{definition}
The fibres over points of the scheme $B$ correspond to out geometric objects, in context of example \ref{exLinesInPlane} the fibres were non-vertical lines in $\mathbb{R}^{2}$.
\par It's usually easier to deal with families when we define a notion of equivalent families. 
That is, an equivalence relation on the set of families, $S(B)$, over the scheme $B$. 
Since we are dealing with schemes, an obvious notion of equivalence between families $\mathcal{X}/B$ and $\mathcal{X}'/B$ can be defined with isomorphisms. That is, an isomorphism, $\psi: \mathcal{X} \to \mathcal{X}'$ such that the following diagram commutes:
\[
\begin{tikzcd}
    {\mathcal{X}} && {\mathcal{X}'}\\
    \\
                  & {B} 
     \arrow["\psi", from = 1-1, to = 1-3]
     \arrow[from = 1-1, to = 3-2]
     \arrow[from = 1-3, to = 3-2]
\end{tikzcd}
\]
Note that isomorphisms over the scheme $B$ is just an example of an equivalence relation on $S(B)$, over the rest of the thesis we will encounter different equivalence realtions which are appropriate for the given situations.
When two families $\mathcal{X}/B$ and $\mathcal{N}/B$ lie in the same equivalence class in $S(B)$, we write $\mathcal{X}/B \simeq \mathcal{N}/B$.
\par Finally, a moduli problem consists of:
\begin{itemize}
    \item A class of geometric objects.
    \item A family of these geometric objects over a parameter space (base scheme).
    \item And an equivalence relation on the set of families, $S(B)$, over a base $B$.
\end{itemize}

An important idea while dealing with families of objects is being able to change your base space.
If we have a family $\mathcal{X}/B$ and a morphism $\varphi: B' \to B$ then one can define a new family over the base $B'$ given by the fibre product $\mathcal{X} \times_{B} B' \to B'$.
\[
\begin{tikzcd}
    {\varphi^{*}\mathcal{X} := \mathcal{X} \times_{B} B'} & & \mathcal{X}\\
    \\
    B' && B
    \arrow[ from = 1-3, to = 3-3]
    \arrow[from = 1-1, to = 1-3]
    \arrow["\varphi", from = 3-1, to = 3-3]
    \arrow[from = 1-1, to = 3-1]
\end{tikzcd}
\]
The family $\varphi^{*}\mathcal{X}/B'$ is called the pullback family of $\mathcal{X}/B$ along the morphism $\varphi$.
For our families to behave ``nicely" we would like pullbacks to commute with equivalence. 
That is, if $\mathcal{X}/B \simeq \mathcal{N}/B $ and $\varphi B' \to B$ is a morphism then we would like:
\begin{align*}
    \varphi^{*}\mathcal{X}/B' \simeq \varphi^{*}\mathcal{N}/B'.
\end{align*}

Before we formally define a moduli problem in the context of categories let's look at example \ref{exLinesInPlane} carefully.
\begin{example}
    \label{compExLinesInplane}
    In example \ref{exLinesInPlane} we defined a parameter space for non-vertical lines in $\mathbb{R}^{2}$. 
    But we would like to define a parameter/base space such that we can characterize all lines (including vertical ones) in $\mathbb{R}^{2}$.
    \par In the diagram above we have embedded $\mathbb{R}^{2}$ in $\mathbb{R}^{3}$ as the plane $z = 1$.
    For each line $ax + by + c = 0 $, there exists a plane, $L := ax + by + cz = 0$,  through the origin in $\mathbb{R}^{3}$ such that $L \cap \{z=1\}$ corresponds to the line $ax + by + c = 0 $.
    \par We also know that each plane through the origin, $\mathbb{R}^{3}$, corresponds to a line through the origin (its normal vector). 
    This gives a bijection between the set of lines in $\mathbb{R}^{2}$ and the points of $\mathbb{RP}^{2}\backslash \{ [0:0:1]\}$, which can be seen as the family:
    \[
        \begin{tikzcd}
            \mathcal{Y} := \{\text{set of lines in } \mathbb{R}^{2}\}\\
            \\
            \mathbb{RP}^{2}\backslash {[0:0:1]}
            \arrow["\psi", from = 1-1, to = 3-1]
        \end{tikzcd}
    \]
    Notice that we are one point away from our parameter space being the entire projective space $\mathbb{RP}^{2}$.
    It's natural to ask what space does $\mathbb{RP}^{2}$ parameterise?
    Luckily the answer to this question is quite striaghtforward.
    We have a bijection between lines in $\mathbb{RP}^{2}$ and points in $\mathbb{RP}^{2}$ given by:
    \[
        \begin{tikzcd}
            {\mathbb{RP}^{2}} & & {\{\text{lines in }\mathbb{RP}^{2}\}} \\
            {[a:b:c]} \arrow[u, phantom, sloped, "\in"]& & {\{[x:y:z] \in \mathbb{RP}^{2}: ax + by + cz = 0 \}}\arrow[u, phantom, sloped, "\in"].
            \arrow["\gamma", leftarrow, from = 1-1, to = 1-3]
            \arrow[ leftrightarrow, from = 2-1, to = 2-3]
            %\arrow[phantom, from = 2-1, to = 1-1, sloped, "\in"]
            %\arrow[phantom, from = 2-2, to = 1-2, sloped, "\in"]
        \end{tikzcd}
    \]
    And this can be seen as a family $\gamma : \mathbb{RP}^{2}\to \mathbb{RP}^{2}$ of lines in $\mathbb{RP}^{2}$.
    \par We can now interpret the families $\mathcal{X}/\mathbb{R}^{2}$ (example \ref{exLinesInPlane}) and $\mathcal{Y}/\{\mathbb{RP}^{2}\backslash {[0:0:1]}\} $ as families of lines in $\mathbb{RP}^{2}$ by embedding $\mathbb{R}^{2}$ in $\mathbb{RP}^{2}$, giving us the three families:
    \begin{itemize}
        \item $\varphi : \mathcal{X} \to \mathbb{R}^{2}$
    \item $\psi: \mathcal{Y} \to \mathbb{RP}^{2}\backslash {[0:0:1]}\} $
    \item $\gamma: \mathbb{RP}^{2} \to \mathbb{RP}^{2}$.
\end{itemize}
For the inclusion maps, 
\begin{itemize}
    \item $i: \mathbb{R}^{2}\hookrightarrow \mathbb{RP}^{2}$,
\item $j: \mathbb{RP}^{2}\backslash {[0:0:1]}\} \hookrightarrow \mathbb{RP}^{2}$,
\end{itemize}
we have: 
\begin{align*}
    i^{*}\mathbb{RP}^{2} \simeq &\mathcal{X}\\
    j^{*}\mathbb{RP}^{2} \simeq &\mathcal{Y}.
\end{align*}
Further, for any family of lines in $\mathbb{RP}^{2}$, $\mathcal{Z}/B$, there exists a unique morphism $\phi :B \to \mathbb{RP}^{2}$, such that $\mathcal{Z}/B$ is equivalent to the pullback family along $\phi$.
We will see that the family $\gamma: \mathbb{RP}^{2} \to \mathbb{RP}^{2}$ satisfies the properties of a \textbf{universal family}, with the base space being defined as a \textbf{fine moduli space}.
\end{example}

\begin{definition}[Fine moduli spaces]
    A \textbf{universal family} for a moduli problem is a family $U/M$ such that for any other family $\mathcal{X}/B$ there exists a unique morphism $\varphi: B \to M$ with the property that $\varphi^{*}U \simeq \mathcal{X}$.
    \par The base space of the universal family $M$ is called a \textbf{fine moduli space}. The morphism $\varphi$ is called the \textbf{classifying map} of the family $\mathcal{X}/B$.
\end{definition}

\subsection{Moduli Problems in the language of Categories}
The properties of a moduli problem, like commutativity of pullback families with equivalences, can be stated elegantly in the language of categories.
To see this, consider the following commutative diagram:
\[
    \begin{tikzcd}
        {\varphi^{*}\mathcal{X} := \mathcal{X} \times_{B} B'} & & \mathcal{X}\\
        \\
        B' && B
        \arrow[ from = 1-3, to = 3-3]
        \arrow[from = 1-1, to = 1-3]
        \arrow["\varphi", from = 3-1, to = 3-3]
        \arrow[from = 1-1, to = 3-1]
    \end{tikzcd}
\]
Given a morphism $\varphi: B' \to B$ we get a map from $S(B)$ to $S(B')$ which sends $\mathcal{X}/B$ to $\varphi^{*}\mathcal{X}/B'$. 
Remember that $S(B)$ denotes the equivalence class of families over $B$ - hence ensuring that pullback is defined upto equivalence.
We can interpret this reversing of directions as a contravariant functor which sends each scheme to the equivalence class of families over it. 
Consider a functor $S$ defined as:
\begin{align*}
    S:\textbf{Sch}^{op} & \to \textbf{Sets}\\
    X &\mapsto S(X):=\{\text{Equivalence classes of families over }X\},
\end{align*}
where for $\varphi \in \text{Mor}(A,B)$, $S(\varphi) = \varphi^{*} \in \text{Mor}(S(B),S(A))$. 
\subsubsection{Representable Funtors and Fine Moduli Spaces}
\begin{definition}[Functor of Points]
    For any scheme $X$ there exists a contravariant set-valued functor, $h_{X}$, called its \textbf{funtor of points}.
    This funtor is defined as: 
    \begin{align*}
        h_{X}: \textbf{Sch}^{op} &\to \textbf{Sets}\\
        B &\mapsto \emph{Hom}(B,X),
    \end{align*}
    where for $\varphi:B' \to B$, $h_{X}(\varphi)$ is defined as:
    \begin{align*}
        h_Y(\varphi):\emph{Hom}(B,X) &\to \emph{Hom}(B',Y)\\
        \beta &\mapsto \beta \circ \varphi
    \end{align*}
\end{definition}
\begin{definition}[Representable Functors]
    A functor $S$ isomorphic to a functor points (for some scheme $X$) is called a \textbf{representable functor}.
    If $\mathcal{U}:h_{Y}\to S$ is an isomrphism of functors, then we say that the pair $(Y, \mathcal{U})$ represents $S$.
\end{definition}
By Yoneda's Lemma (appendix) we knoww that we can identify the natiral transformation $\mathcal{U}:h_{Y}\to S$ with an element of $S(Y)$.
Suppose the element of $S(Y)$ corresponding to $\mathcal{U}$ is $U$.
Then $(Y,U)$ is said to represent $S$.
\begin{proposition}
    A family $U/M$ is the universal family (with M the fine moduli space) for a moduli problem, if and only if the pair $(M,U)$ represents the moduli functor $S$.
\end{proposition}
\begin{proof}
    \textit{Suppose $(M,U)$ represents the moduli functor $F$}.
    \par Then there exists an isomorphism of functors:
    \begin{align*}
        \mathcal{U}:h_{M} \to F,
    \end{align*}
    which at the level of objects maps in the following manner:
    \begin{align*}
        \mathcal{U}_{M}:h_{M}(M) &\to F(M)\\
        \text{id}_{M} &\mapsto U/M.
    \end{align*}
    Let $\mathcal{V}:= \mathcal{U}^{-1}$ be the inverse functor. Now consider any $\varphi: B \to M$, we have the commutative diagram:
    \[
        \begin{tikzcd}
            h_{M}(M) & & F(M) & & h_{M}(M)\\
            h_{M}(B) & & F(B) & & h_{M}(B)
            \arrow["\mathcal{U}_{M}", from = 1-1, to = 1-3]
            \arrow["\mathcal{V}_{M}", from = 1-3, to = 1-5]
            \arrow["\mathcal{U}_{B}", from = 2-1, to = 2-3]
            \arrow["\mathcal{V}_B", from = 2-3, to = 2-5]
            \arrow["F(\varphi)", from = 1-3, to = 2-3]
            \arrow["\_\circ \varphi", from = 1-1, to = 2-1]
            \arrow["\_\circ \varphi", from = 1-5, to = 2-5]
        \end{tikzcd}
    \]
    Since $\mathcal{U}$ is an isomorphism $\mathcal{U}_{B}$ is a bijection for all $B$. Implying that for all $\mathfrak{X}/B \in F(B)$ there exists $\varphi_{\mathfrak{X}}\in h_{M}(B)$ such that $\mathcal{U}_{B}(\varphi_{\mathfrak{X}}) = \mathfrak{X}/B$. 
    For this morphism, $\varphi_{\mathfrak{X}}$, consider the following diagram (with the element being chased on the right).
    \[
        \begin{tikzcd}
            h_{M}(M) & & F(M) & & \text{id}_{M} && U/M\\
            h_{M}(B) & & F(B) & & \varphi_{\mathfrak{X}} && \mathfrak{X}/B = F(\varphi_{\mathfrak{X}})(U/M) = \varphi_{\mathfrak{X}}^{*}U/B
            \arrow["\mathcal{U}_{M}", from = 1-1, to = 1-3]
            \arrow["\mathcal{U}_{B}", from = 2-1, to = 2-3]
            \arrow["F(\varphi_{\mathfrak{X}})", from = 1-3, to = 2-3]
            \arrow["\_\circ \varphi_{\mathfrak{X}}", from = 1-1, to = 2-1]
            \arrow["\mathcal{U}_{M}", mapsto, from = 1-5, to = 1-7]
            \arrow["\mathcal{U}_{B}", mapsto, from = 2-5, to = 2-7]
            \arrow["\_\circ \varphi_{\mathfrak{X}}", mapsto, from = 1-5, to = 2-5]
            \arrow["F(\varphi_{\mathfrak{X}})", mapsto, from = 1-7, to = 2-7]
        \end{tikzcd}
    \]
    Via commutativity of the above diagram it follows that $\mathfrak{X}/B = \varphi_{\mathfrak{X}}^{*}U/B$. 
    The classifying map, $\varphi_{\mathfrak{X}}$, is unique since the $\mathcal{U}_{B}$ is a bijection. 
    Hence, we have proved one direction of the hypothesis.
    \par \textit{Suppose $U/M$ is a universal family}.
    \par We need to show that the naturam transformation $\mathcal{U}:h_{M} \to F$, which corresponds to $U/M \in F(M)$ (via the bijection from Yoneda's lemma), is an isomorphism of functors.
    Recall that a natural transformation, $\mathcal{U}$, is an isomorphism if for a each $B \in \textbf{Sch}$ the induced map $\mathcal{U}_{B}$ is a set theoretic bijection. 
    We showed above that for any $\varphi \in h_{M}(B)$, $\mathcal{U}_{B}(\varphi) = \varphi^{*}U/B$: 
    \begin{align*}
        \mathcal{U}_{B}: h_{M}(B) = \text{Hom}(B,M) &\to F(B) \\
        \varphi &\mapsto \varphi^{*}U/B.
    \end{align*}
    Further, via the universal propety of $U/M$, for each family $\mathfrak{X}/B$ there exists a unique morphism (the classifying map) $\varphi_{\mathfrak{X}} \in h_{M}(B)$. 
    Let this correspondence between families over $B$ and morphisms in $\text{Hom}(B,M)$ by a map $L$. Then,
    \begin{align*}
        L \circ \mathcal{U}_{B} (\varphi) &= L(\varphi^{*}U/B) = \varphi \\
        \mathcal{U}_{B} \circ L (\mathfrak{X}/B) &= \mathcal{U}_{B} (\varphi_{\mathfrak{X}}) = \varphi^{*}_{\mathfrak{X}} U/M = \mathfrak{X}/B .
    \end{align*}
    Therefore, $\mathcal{U}_{B}$ is a bijection for all $B \in \textbf{Sch}$, proving that $\mathcal{U}$ is an isomorphism of functors.
\end{proof}

\subsubsection{Coarse Moduli Spaces}

Notice that the previous proposition doesn't deal with the existence of fine moduli spaces in general.
There are many interesting moduli functors (one which we will encounter later) in the thesis which aren't representable.
That is, fine moduli spaces need not exist for such moduli problems.
It's now natural to ask if there is a way we can relax the definition of a fine moduli space to construct some sort of a \textit{best approximation}? We will call this approximation the \textbf{coarse moduli space}.

\begin{definition}
    A \textbf{coarse moduli space} for a moduli functor $S$ is the pair $(M, \mathcal{V})$, where $M$ is a scheme and $\mathcal{V}:S \to h_{M}$ is a natural transform of functor such that:
    \begin{itemize}
        \item $(M,\mathcal{V})$ is initial across all such pairs,
        \item The map between sets, 
            \begin{align*}
                \mathcal{V}_{\emph{Spec}\mathbb{C}}: S(\emph{Spec}\mathbb{C}) \to \emph{Hom}(\emph{Spec}\mathbb{C}, M),
            \end{align*}
            is a bijection.
    \end{itemize}
\end{definition}
\subsection{Families of Points in $\mathbb{P}^{1}$}
In this section we will look at families of distint point in $\mathbb{P}^{1}$. 
This example will serve as an important building block for the study of moduli spaces of marked rational plane curves. 
For the rest of this section we denote with $x$, the point $[x:1] \in \mathbb{P}^{1}$, and $\infty$ for the point $[0:1]$.
\par By a family of $n$ points in $\mathbb{P}^{1}$ over a scheme $B$ we mean a diagram:
\[
    \begin{tikzcd}
        B \times \mathbb{P}^{1}\\
        \\
        B
        \arrow["\pi",xshift = -2ex, from = 1-1, to = 3-1]
        \arrow[ xshift = 1ex,from = 3-1, to = 1-1]
        \arrow[ xshift = 2.3ex,from = 3-1, to = 1-1]
        \arrow[ swap, "\sigma_{i}", xshift = 3.6ex,from = 3-1, to = 1-1]
    \end{tikzcd}
\]
Where $\sigma_{i}$ are disjoint sections. Therefore for each $b \in B$ we get $n$ points in $\mathbb{P}^{1}$ given by the $n$ sections. 
\par For the rest of this section we are interested in the case where $n \geq 4$, since for any three distinct points, $p_{1}, p_{2}, p_{3}$, there exists an automorphism of $\mathbb{P}^{1}$ which sends them to $0,\,1,\text{ and }\infty$ respectively - implying all tuples of three or less points are equivalent.

\subsubsection{Projective Equivalence of Families}
Two families of $n$-points over a scheme $B$ are said to be projectively equivalent if there exists an automorphism $\phi \in \text{Aut}(\mathbb{P}^{1})$ such that the diagram:
\[
    \begin{tikzcd}
        B \times \mathbb{P}^{1} & & B \times \mathbb{P}^{1}\\
                                & & \\
        B & & B
        \arrow["\pi",xshift = -2ex, from = 1-1, to = 3-1]
        \arrow[ swap, "\sigma_{i}", xshift = 1.2ex,from = 3-1, to = 1-1]
        \arrow["\pi",xshift = -2ex, from = 1-3, to = 3-3]
        \arrow[ swap, "\sigma_{i}'", xshift = 1.2ex,from = 3-3, to = 1-3]
        \arrow["\text{id}\times \phi", from = 1-1, to = 1-3]
        \arrow[equal,  from = 3-1, to= 3-3]
    \end{tikzcd}
\]
commutes.

\subsubsection{Family of 4 points - Quadruples}
Define $Q:= \mathbb{P}^{1} \times \mathbb{P}^{1} \times \mathbb{P}^{1} \times \mathbb{P}^{1} \backslash \{\text{diagonals}\}$. 
Then for any family of four points over a scheme $B$ we have a map $\sigma: B \to Q$ where each coordinate is defined by the section $\sigma_{i}$.
This map $\sigma$ acts as the classifying map, and shows that $Q$ is a fine moduli space for families of four points and the universal family is of the form:
\[
    \begin{tikzcd}
        Q \times \mathbb{P}^{1}\\
        \\
        Q
        \arrow["\pi",xshift = -2ex, from = 1-1, to = 3-1]
        \arrow[ xshift = 1ex,from = 3-1, to = 1-1]
        \arrow[ xshift = 2.3ex,from = 3-1, to = 1-1]
        \arrow[ xshift = 3.6ex,from = 3-1, to = 1-1]
        \arrow[ swap, "\sigma_{i}", xshift = 4.9ex,from = 3-1, to = 1-1]
    \end{tikzcd}
\]
with $\sigma_{i}(\textbf{p}) = (\textbf{p},p_{i})$.
\par Now we want to take into consideration the notion of projective equivalence. 
Remember that any three points $p_{1},\, p_{2}, \,p_{3}$ define an automorphism $\alpha$ of $\mathbb{P}^{1}$ which sends them to $0,\,1\,\text{and }\infty$.
For a quedrupule of points $p_1, \dots, p_4$, let $\alpha$ be the automorphism determined by the first three points. 
Then we define the \textbf{cross ratio} of the quadruple as $\alpha(p_{4}) \in \mathbb{P}^{1}\backslash\{0,1,\infty\}$.
\par It follows that each quadruple $(p_{1}, \dots, p_{4})$ is projctively equivalent to $(0,1,\infty,q)$ where $q$ is the cross ratio of the quadruple. Consquently we have a bijection:
\begin{align*}
    \mathcal{M}_{0,4} := \{\text{Equivalence class of quadruples}\} \leftrightarrow \mathbb{P}^{1}\backslash\{0,1,\infty\}.
\end{align*}
It's easy to show that $\mathcal{M}_{0,4}$ is the \textbf{fine moduli space} for families of quadruples with the universal family given by:
\[
    \begin{tikzcd}
        \mathcal{M}_{0,4} \times \mathbb{P}^{1}\\
        \\
        \mathcal{M}_{0,4},
        \arrow["\pi",xshift = -2ex, from = 1-1, to = 3-1]
        \arrow[ xshift = 1ex,from = 3-1, to = 1-1]
        \arrow[ xshift = 2.3ex,from = 3-1, to = 1-1]
        \arrow[ xshift = 3.6ex,from = 3-1, to = 1-1]
        \arrow[ swap, "\tau_{i}", xshift = 4.9ex,from = 3-1, to = 1-1]
    \end{tikzcd}
\]
where $\tau_1 :=0$, $\tau_2 :=1$, $\tau_3 := \infty$, $\tau_4(p) = p$.

\subsubsection{Family of n-points}
For a quadruples of points $(p_{1}, p_{2}, p_{3}, p_{4})$ let $\lambda((p_{1}, p_{2}, p_{3}, p_{4}))$ denote their cross ratio.
Then we can sharacterize projective equivalence of two $n$-tuples in the following manner.
Let $(p_{1}, p_{2}, p_{3},\dots,p_{n})$ and $(p'_{1}, p'_{2}, p'_{3},\dots,p'_{n})$ be two $n$-tuples; they are said to be projectively equivalent if and only if for all $i\geq 4$:
\begin{align*}
     \lambda(p_{1}, p_{2}, p_{3}, p_{i}) = \lambda(p'_{1}, p'_{2}, p'_{3}, p'_{i}).
\end{align*}
With this characterisation it's easy to see that:
\begin{align*}
    \mathcal{M}_{0,n} \simeq \mathcal{M}_{0,4} \times \dots \times \mathcal{M}_{0,4} \backslash\{diagonals\},
\end{align*}
where the product has $n-3$ elements.

\section{Stable n-pointed Curves}
For the rest of this thesis, a rational plane curve (or a plane curve) will refer to a one dimensional variety that's birational to $\mathbb{P}^{1}$. 
In this section we will look at moduli spaces of plane curves with $n$-marked points.
This will help us in building intution for understanding the moduli space of stable maps, which we will use to prove Kontsevich's forumla.
A plane curve, C, with $n$ distinct point $p_{1},\dots, p_{n} \in C$ will be called a $n$-pointed rational curve, and will be denoted by the tuple $(C, p_{1}, \dots, p_{n})$.

\subsection{Families of smooth $n$-pointed curves}
\begin{definition}
    A family of $n$-pointed rational plane curves is a flat and proper morphism of schemes $\pi: X \to B$, such that the geometric fibres $\pi^{-1}(b)$ for all $b \in B$ are projective rational curves.
    Further, this morphism is equipped with $n$ disjoint sections $\sigma_{i}:B \to X$ which represent the $n$ marked points on each fibre.
\end{definition}
An equivalence between two families $\pi: \mathfrak{X} \to B$ and $\pi': \mathfrak{X}' \to B$ is an isomorphism $\phi: \mathfrak{X} \to \mathfrak{X}'$ such that the following diagram commutes:
\[
    \begin{tikzcd}
        \mathfrak{X} & & \mathfrak{X}'\\
                                & & \\
        B & & B
        \arrow[swap, "\pi",xshift = -1ex, from = 1-1, to = 3-1]
        \arrow[ swap, "\sigma_{i}", xshift = 1ex,from = 3-1, to = 1-1]
        \arrow[swap, "\pi",xshift = -1ex, from = 1-3, to = 3-3]
        \arrow[ swap, "\sigma_{i}'", xshift = 1ex,from = 3-3, to = 1-3]
        \arrow["\phi", from = 1-1, to = 1-3]
        \arrow[equal,  from = 3-1, to= 3-3]
    \end{tikzcd}
\]
For families, $\mathfrak{X}/B$ where the geomteric fibres are \textit{isomorphic} to $\mathbb{P}^{1}$:
\begin{itemize}
    \item If $n=1$ then $\mathfrak{X} \simeq \mathbb{P}(\varepsilon)$, where $\varepsilon$ is a rank 2 vector bundle on $B$.
    \item For $n=2$ this vector bundle splits.
    \item And for $n\geq 3$ there is a unique isomorphism $\mathfrak{X}\to B \times \mathbb{P}^{1}$. 
        Which makes this case equivalent to classification of families of $n$ points on $\mathbb{P}^{1}$ upto projective equivalence.
\end{itemize}
From the third point above it follows that, for $n\geq 4$, there exists a fine moduli space $\mathcal{M}_{0,n}$ for the problem of classifying $n$-pointed \textit{smooth} rational curves up to isomorphism.
We have seen that for $n=3$, $\mathcal{M}_{0,3}$ is a single point space, hence are first interesting example is $\mathcal{M}_{0,4}$ which can be interpreted as $\mathbb{P}^{1}\backslash\{0,1,\infty\}$.

\subsection{$\mathcal{M}_{0,4}$ and Compactification of $\mathcal{M}_{0,n}$}
When we see $\mathcal{M}_{0,4}$ as the projective line minus three points it's clear that it is not a compact space.
In general, for a good intersection theory to exist on our moduli space, it is necessary for them to be compact. 

\begin{example}
    \label{M04Example}
    We look at the universal family over $\mathcal{M}_{0,4}$ and see what happens when we compactify it to $\mathbb{P}^{1}$. 
    As discussed in the previous section, the universal family is $\pi :\mathcal{M}_{0,4} \times \mathbb{P}^{1} \to \mathcal{M}_{0,4}$, with sections $\tau_1 :=0$, $\tau_2 :=1$, $\tau_3 := \infty$, $\tau_4(p) = p$.
    This can be visualised as the following diagram, where $U_{q} = \pi^{-1}(q)$
    \begin{center}
    \includegraphics[scale = 0.3]{Chapters/Images/M_04_UniversalFam.png}
    \end{center}
    If we compactify $\mathcal{M}_{0,4}$ to $\mathbb{P}^{1}$ then the sections will no longer be disjoint.
    The diagnoal section $\tau_{4}$ will intersect the other $\tau_{i}'$s at $0,\,1\,\text{and }\infty$ respectively. 
    Therefore, the geometric sections over the points $0,\,1,\, \text{and }\infty$ are $3$-pointed curves, rather than $4$ pointed ones.
    \par To fix this problem we can blow up the universal family $\mathbb{P}^{1} \times \mathbb{P}^{1}$ at the three points, $\tau_{4}\cap \tau_{i}$, for $1\leq i \leq 3$.
    This process of blowing replaces each point $\tau_{4}\cap \tau_{i}$ with a copy of $\mathbb{P}^{1}$ (the exceptional divisor).
    Let $\tilde{\tau}_{i}$ denote the sections to the the blowup of the unversal family, then the fibre over $0$, $U_{0}$, looks like the following.
    \begin{center}
    \begin{tikzpicture}
        \draw (1,0) -- (1,4);
        \draw (0,1) -- (4,0);
    \draw[Circle-Circle] (1,3) node[right] {$\tilde{\tau}_{3}(0)$} -- (1,2)node[right] {$\tilde{\tau}_{2}(0)$};
        \draw[Circle-Circle] (2,0.5)node[below] {$\tilde{\tau}_{1}(0)$} -- (3,0.25)node[below] {$\tilde{\tau}_{4}(0)$};
    \end{tikzpicture}    
    \end{center}
    This fibre clearly isn't a smooth variety isomorphic to $\mathbb{P}^{1}$, but it gives us an idea of curves we may want to include in our family if we want our moduli space to be compact. 
    To be specific, we will include reducible curves with some additional structures in our family.
\end{example}
Taking the preceeding example as motivation we introduce the notion of \textit{trees} and \textit{stable curves}.
\begin{definition}
    A \textbf{tree of projective lines} is a connected curve such that:
    \begin{itemize}
        \item All irreducible components are isomorphic to $\mathbb{P}^{1}$.
        \item The points of intersection between irreducible components are ordinary double points (so no intersections are tangential).
        \item There are no closed cricuits, that is, if a node is removed, then the curve becomes disconnected.
    \end{itemize}
    We will refer to these as just trees.
    Further, the irreducible components are called \textbf{twigs}.
\end{definition}
\begin{definition}
    Let $n  \geq 3$. 
    We refer to the marked points and nodes as \textit{special points}.
    A stable $n$-pointed rational curve is a tree $C$ of projective lines, with $n$ distinct marked points (which don't overlap with the nodes) such that each twig has atleast $3$ special points. 
\end{definition}

Notice that the fibres over $0,\,1,\,\infty$ in Example \ref{M04Example} are stable $4$-pointed curves with two twigs.

\begin{remark}
 \par The stability condition in the previous definition is equivalent to saying that curve is automorphism free.
Any automorphism of a $n$-pointed stable curves sends the marked point to itself. 
Therefore, any twig with one node will be sent to itself. 
Since singular points are mapped to singular points, the node must be sent to itself to himself.
By induction it's easy to see that all nodes will mapped to themselves, and consquently, all twigs are sent to itself.
This implies that an automorphism of a stable curve is formed by gluing together automorphisms of the twigs.
But by stability, each twig has more than three special points, each of which is mapped to themselves, implying that their automorphisms are trivial.
Showing that the automorphisms for stable curves are trivial.
\end{remark}
\subsection{Forgetting points, Stabilisation and Contraction}
\textcolor{red}{Will add later}
\subsection{$\mathcal{M}_{0,n}$ and Boundary Divisors}

\section{Stable Maps}
\section{Kontsevich's Formula}
