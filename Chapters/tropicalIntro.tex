\section{Introduction and Examples}
\subsection{Tropical Arithmetic}
    \begin{definition} The tropical semiring $(\mathbb{R} \cup \{\infty\}, \oplus, \odot)$, is defined as the semiring whose underlying set is $\mathbb{R}$ with the binary operations: 
        \begin{align*}
            x \oplus y &= \text{min}(x,y)\\
            x \odot y  &= x + y
        \end{align*}
    We refer to $\oplus$ as the \textit{tropical sum}, and $\odot$ as the \textit{tropical product}. This can be a little confusing since the tropical product is defined as out usual notion of sum.
    \end{definition}

\subsection{Tropical plane curves}
\subsection{Valuations and Varieties}
    \begin{definition}
        For a field $K$ we define the valuation $\emph{val}$ as a function $\emph{val}: K \to \mathbb{R}\cup \{\infty\}$ such that:
        \begin{enumerate}
            \item $\emph{val}(a) = \infty \Leftrightarrow a = 0$
            \item $\emph{val}(ab) = \emph{val}(a)+ \emph{val}(b)$
            \item $\emph{val(a+b)} \geq min(\emph{val}(a),\emph{val}(b))$
        \end{enumerate}
        The image of $\emph{val}$ is an additive subgroup $\Gamma_{\emph{val}} \subset \mathbb{R}$, called the value group.
    \end{definition}

    \begin{lemma}
        \label{vallemma}
        When $\text{val}(a) \neq \text{val}(b)$ for $a,b \in K$, then $\text{val}(a+b)= \text{min}(\text{val}(a),\text{val}(b))$.
    \end{lemma}

    \begin{lemma}
        Suppose $K$ is an algebraically closed valued field, then the value group $\Gamma_{\emph{val}}$ is divisible and dense in $\mathbb{R}$.
    \end{lemma}

    \begin{lemma}
        When $K$ is an algebraically closed valued field, the surjection $K^{*} \twoheadrightarrow \Gamma_{\emph{val}}$ splits.
    \end{lemma}

    \begin{proposition}
        For $I \subset K[x_{1}^{\pm 1},\dots,x_{n}^{\pm 1}]$ let $X = V(I) \subset T^n$, then $V(I_{\text{aff}}) = \overline{i(X)}$.
    \end{proposition}

    \begin{proposition}
        \label{densenessprop}
        $(K,v)$ be a valued field with splitting $\Gamma_{\emph{val}} \twoheadrightarrow \mathbb{R},~w \to t^{w}$.
        Let $\alpha_{1}, \dots,\alpha_{n} \in \tilde{K}^{*}$ and $w_1, \dots, w_n \in \Gamma_{\emph{val}}$. 
        Conisder the set of $\textbf{y} = (y_1, \dots, y_n) \in T^{n}$ such that $\emph{val}(y_{i}) = w_{i}$ and $\overline{t^{-w_i}y_i} = \alpha_{i}$ for all $1\leq i \leq n$.
        This set is dense in $T^n$ with respect to the Zariski topology.
    \end{proposition}
\subsection{Polyhedral Geometry}

Section $2.3$ \cite{maclagan2015introduction} concisely summarises the requisite polyhedral geometry pretty well. We will only discuss the construction of the regular subdivision in this report.
   Before moving onto the theory of groebner basis and tropical varieties we will define some basic notions from polyhedral geometry.
   \par A set $X\subset \mathbb{R}^{n}$ is said to be \textbf{convex} if for any two points in $X$ the line connecting those two points also lies in $X$. 
   The \textbf{convex hull}, conv$(U)$, of a set $U\subset \mathbb{R}^{n}$ is the smallest convex set containg $U$.
   When $U$ is a finite set $\text{conv}(U)$ is a \textbf{polytope}. 
   A \textbf{polyhedral cone} $C \subset \mathbb{R}^{n}$ is the \textit{positive hull} of a finite subset of $\mathbb{R}^{n}$. 
   For a set $U = \{u_1,\dots,u_n\}$:
   \begin{align*}
       C = \text{pos}(U) = \left\{\sum_{i=1}^{n}\lambda_i u_i~:~\lambda_i\geq 0\right\}
   \end{align*}
   When the $u_i$'s are linearly independent, the cone is said to be \textbf{simpilicial}. 
    It's easy to see that every polyhedral cone is a finite intersection of \textit{half n-spaces}, this perspective allows to define cones as the set $\{x \in \mathbb{R}^{n}: Ax\leq 0\}$, where $A$ is a $d \times n$ matrix. 
    For a dual vector $w \in (\mathbb{R}^{n})^{\vee}$ we define the associated face in the cone $C$ as:
    \begin{align*}
        \text{face}_{w}(C) = \left\{x \in C: w\cdot x \leq w\cdot y ~\text{for all}~y \in C\right\}
    \end{align*}
    For $d'$ the dimension of the face, there also exists a $d'\times n$ matrix $A'$ such that $\{x \in C: A'x\ =  0\}$. 
    A \textbf{polyhedral fan} is a collection of polyhedral cones such that every face of the cone is part of the collection, and the intersection of any two cones is a face of both the cones.
    A \textbf{polyhedron} $P \subset \mathbb{R}^{n}$ is the intersection of finitely many affine half spaces, allowing us to define it as the set $\{x \in \mathbb{R}^{n}: Ax\leq b\}$ where $b \in \mathbb{R}^{d}$ and $A$ is a $d \times n$ matrix. We can define faces of polyhedron's in the same manner as we defined faces of a cone.
    \par A \textbf{polyhedral complex}, $\Sigma$, is a collection of polyhedron such that: if $P \in \Sigma$ then all faces of $P$ also lie in $\Sigma$, and if $P,Q \in \Sigma$ then $P \cap Q = \emptyset$ or $P \cap Q$ is a face of both $P,Q$. The support of a polyhedral complex, $|\Sigma |$ is the underlying set of points in $\Sigma$. 
    The \textbf{lineality space} of a polyhedral complex, if it exists, is the largest linear subspace which lies in it.
    For any set of points in $\mathbb{R}^{n}$, their \textbf{affine span} is defined as the smallest affine subspace containing those points.

    The \textbf{regular subdivision} is an important construction since it helps us create a link between the newton polytope and tropical variety. 
    \begin{definition}
    For an ordered list of vectors $v_1,\dots, v_r \in \mathbb{R}^{n}$, fix a weight vector $w = (w_1,\dots,w_n) \in \mathbb{R}^{n}$. 
    The \textit{regular subdivision} is the polyhedral fan with support $\text{pos}(v_1,\dots,v_n)$ whose cones are $\text{pos}(\sigma)$ for all $\sigma \subset \{v_1,\dots,v_n\}$ such that there exists $c \in \mathbb{R}^{n+1}$ for which:
    \begin{itemize}
        \item $c \cdot v_i = w_i$ for all $v_{i}\in \sigma$ 
        \item $c \cdot v_i < w_i$ for all $v_i \not\in \sigma$
    \end{itemize}
    Although defined as a fan, regular subdivisions are usually defined for polyhedral complexes.
    \end{definition}
    Suppose you have a polyhedron $P$ given by $\text{conv}(u_i: 1\leq i\leq r)$, then we look at the vectors $v_i = (u_i,1)$. The regular subdivision of $v_1,\dots,v_r$ with respect to a weight vector $w = (w_1, \cdots, w_r)$ can be interpreted as projection of the bottom faces of the polytope $P_w = \text{conv}\{(u_i,w_i): 1\leq i\leq r\}$. That is in the polytope $P_w$ consider the faces $F$ whose normal vector has positive final coordinates and project them onto the first $r$ coordinates, this gives the regular subdivision of $P$ with respect to $w$.


\section{Tropical Hypersurfaces}

    \begin{definition}
        Consider polynomial $f = \sum c_u x^u \in K[x_0,\dots,x_n]$, and let
        \[
            W = \text{trop}(f)(w) = \text{min}_{u \in \mathbb{Z}^{n+1}}\{\text{val}(c_u) + u\cdot w\}.
        \]
        Then we define the initial form of $f$ with respect to $w$ as
        \begin{align*}
            \text{in}_{w}(f)(x) = \sum_{\text{val}(c_u) + u\cdot w = W} \overline{c_u t^{-\text{val}(c_u)}} x^u.
        \end{align*}
        We can also define the initial form for a laurent polynomial $f$ in the same way.
    \end{definition}

    It's easter to discuss properties of these initial forms once we understand their relation to tropical hypersurfaces.
    So before we further study these intial forms, we will first look at tropical hypersurfaces.
    \begin{definition}
        For a laurent polynomial $f \in  K[x_{1}^{\pm1}, \dots, x_{n}^{\pm1}] $, the tropical hypersurface $\emph{trop}(V(f))$ is the set:
        \begin{align*}
        \{w \in \mathbb{R}^{n}: \text{minimum in }\emph{trop}(f)(w)\text{ is acheived atleast twice}\}.
        \end{align*}
    \end{definition}
    Similarly, for a tropical polynomial $F$, $V(F)$ denotes the set:
    \begin{align*}
        \{w \in \mathbb{R}^{n}: \text{minimum in }F(w)\text{ is acheived atleast twice}\}.
    \end{align*}
    It's clear that for a laurent polynomial $f$: $V(\text{trop}(f)) = \text{trop}(V(f))$

\subsection{Kapranov's Theorem}
In this section we will discuss Kapranov's Theorem, which gives various characterisations for a tropical hypersurface.
    \begin{theorem}[Kapranov's Theorem]
    Let $K$ be an algebraically closed field with a non-trivial valuation. Fix a laurent polynomial $f \in K[x_{1}^{\pm1}, \dots, x_{n}^{\pm1}]$. The following sets are the same:
    \begin{enumerate}
        \item $A$ = $V(\text{trop}(f))$.
        \item $B$ = $\{w \in \mathbb{R}^{n}: \text{in}_{w}(f)~ \text{is not a monomial}\}$.
        \item $C$ = closure of $\left\{(\text{val}(y_1), \dots, \text{val}(y_n)): (y_1,\dots, y_n) \in V(f)\right\}$ in $\mathbb{R}^{n}$.
    \end{enumerate}
    Further, if $f$ is irreducible and $w \in \Gamma_{\text{val}} \cap \text{trop}(V(f))$, then the set $\{y \in V(f): \text{val}(y) = w\}$ is Zariski dense in $V(f)$.
    \end{theorem}

    \begin{proof}[Sketch of proof]
        Showing $A$ = $B$ is straightforward. 
        If $x \in A$ then there exist atleast two $u_1,~u_2$ such that $\text{trop}(f)(x) = \text{val}(c_{u_{1}}) + u_{1}\cdot x = \text{val}(c_{u_{2}}) + u_{2}\cdot x$. 
        This implies $\text{in}_{x}(f)$ has atleast the terms corresponding to $u_1$ and $u_2$, implying it's not a monomial, and $A \subset B$.
        Now consider a $w \in B$, the initial form $\text{in}_{w}f = \sum_i \text{val}(c_{u_{i}}) + u_i \cdot w$ has atleast two terms, say for $i=1,2$. 
        Therefore $\text{trop}(f)(w) = \text{val}(c_{u_{1}}) + u_{1}\cdot w = \text{val}(c_{u_{2}}) + u_{2}\cdot w$, implying $w \in A$.
        Hence A = B.
        \par To show $C \subset A$, since $A$ is already closed, we just need to show that the set $\{(\text{val}(y_1), \dots, \text{val}(y_n)): (y_1,\dots, y_n) \in \mathbb{R}^{n}\}$ lies in $A$. Say $\textbf{y} = (y_1,\dots,y_n) \in V(f)$, this implies 
        \begin{align*}
            f(\textbf{y}) &= \sum_u c_u y^{u} = 0 \\
            \Rightarrow&\text{val}(\sum_u c_u y^{u}) = \infty \\
            \Rightarrow& \text{min}_u(\text{val}(c_u) + u \cdot \text{val(\textbf{y})}) = \infty
        \end{align*}
        For this equality to hold, for any pair $u_{1},~u_{2}$ with $c_{u_{i} \neq 0}$ we need to have $\text{val}(c_1) = \text{val}(c_2)$, else by [\ref{vallemma}] $\text{val}(f(\textbf{y})) < \infty$, which is a contradiction. 
        \par To finish of the proof, we show $B \subset C$ and the density statement in the next propostion.
    \end{proof}
    This proposition uses notation from the statement of Kapranov's Theorem.
    \begin{proposition}
        Fix a laurent polynomial $f \in  K[x_{1}^{\pm1}, \dots, x_{n}^{\pm1}]$, and let $w \in \Gamma_{\text{val}^{n}}$. 
        Suppose $w \in B$ and $\alpha \in (\tilde{K}^{*})^{n}$ satisfies $\text{in}_{w}(f)(\alpha) = 0$. 
        There exists $\textbf{y} \in (K^{*})^{n}$ such that $f(\textbf{y}) = 0$,  $\text{val}(\textbf{y}) = \textbf{w}$, and $t^{-1_{i}}y_{i}=\alpha_{i}$ for $1 \leq i \leq n$.
        Further, if $f$ is irreducible, then the set of such $\textbf{y}$ is Zariski dense in the hypersurface $V(f)$.
    \end{proposition}
    \begin{proof}
        We go about proving this proposition via induction on $n$.
        \par \textit{Base Case.} Consider $n = 1$.
        \par We can multiply by a unit and assume that $f = \sum_{i=0}^{s}c_{i}x^{i}$ such that $c_{0}, c_{s}\neq0$. 
        Since we are working in an algebraically closed field $K$, $f = \prod^{s}_{j=1} (a_{j}x-b_{j})$. 
        From \textcolor{red}{[cite lemma from next seciton]}, $\text{in}_{w}(f) = \prod^{s}_{j=1} \text{in}_{w}(a_{j}x-b_{j})$. 
        Since $\alpha \in \tilde{K}$ and $\text{in}_{w}f(\alpha)=0$, there exists $j$ such that $\text{in}_{w}(a_{j}x-b_{j})(\alpha) = 0$.
        Futher, since $w \in B$, the initial form $\text{in}_{w}(f)$ is not a monomial.
        This implies that $\text{in}_{w}(a_{j}x-b_{j})$ is not a monomial, hence:
        \begin{align*}
            \text{val}(a_{j}) + w &= \text{val}(b_{j})\\
            \alpha &= \overline{t^{-w}b_{j}/a_{j}}.
        \end{align*}
        Setting $y = b_{j}/a_{j}$ we get $f(y) = 0$, and $\overline{t^{-\text{val}(y)}y} = \alpha$, proving the base case.
        \par \textit{Inducting on n}. Consider $n>1$.       
        \par Let $\phi^{*}_{l}$ be the automorphism on $ K[x_{1}^{\pm1}, \dots, x_{n}^{\pm1}]$ from the change of variables lemma \textcolor{red}{[cite change of var lemma]}.
        If $\textbf{y}\in T^{n}$ satisfies:
        \begin{align*}
            \phi^{*}_{l}(f)(\textbf{y}) &= 0\\
            \text{val}({y}_{i}) &= w_{i} - l^{i}w_{n} \\
            \overline{t^{-w_{i}- l^{i}w_{n}}y_{i}} &= \alpha_{i}\alpha_{n}^{-l^{i}},
        \end{align*}
        then for  $\textbf{y}'\in T^{n}$, defined by $y'_{i}=y_{i}y_{n}^{l^{i}}$ for $1\leq i<n$, and $y'_{n}  = y_{n}$:
        \begin{align*}
            f(\textbf{y}') &= 0\\
            \text{val}({y'}_{i}) &= w_{i} \\
            \overline{t^{-w_{i}}y'_{i}} &= \alpha_{i}.
        \end{align*}
        Therefore we can replace $f$ with $\phi^{*}_{l}(f)$.
        From \textcolor{red}{lemma} it follows that when $f$ is regarded as a polynomial in $x_{n}$, all coefficients are monomials in $ K[x_{1}^{\pm1}, \dots, x_{n-1}^{\pm1}]$.
        \par Consider the set $\mathcal{V} := \{(y_{1},\dots, y_{n-1})\in T^{n-1}: \text{val}(y_{i}) = w_{i},\,\overline{ t^{-w_{i}}y_{i}} = \alpha_{i} \}$.
        From proposition \ref{densenessprop}, we know that the set $\mathcal{Y}$ is Zariski dense in $T^{n-1}$.
        Additionaly, for each such point $\textbf{y}\in \mathcal{Y}$ the polyomial defined by $g(x_{n}) = f(y_{1},\dots, y_{n-1}, x_{n})$ is non-zero.
        For $\textbf{u} \in \mathbb{Z}^{n}$ we denote with $\textbf{u}'\in \mathbb{Z}^{n-1}$ the projection of $\textbf{u}$ onto the first $n-1$ coordinates. 
        \par With $\textbf{u} = (\textbf{u}',i)$, the polynomial $g$ has the form,
        \begin{align*}
            g(x_{n}) = \sum_{i} c_{\textbf{u}}\textbf{y}^{\textbf{u}'}x_{n}^{i}
        \end{align*}
        And for each monomial corresponding to $\textbf{u}$ we have:
        \begin{align*}
            &~ \text{val}(c_{\textbf{u}}) + \text{val}(\textbf{y}^{\textbf{u}'}) + w_{n}\cdot i\\
            =& ~\text{val}(c_{\textbf{u}}) + \textbf{w} \cdot \textbf{u}.
        \end{align*}
        Therefore, $\text{trop}(g)(w_{n}) = \text{trop}(f)(\textbf{w})$, implying:
        \begin{align*}
            \text{in}_{w_{n}}(g)(x_{n}) =& \sum_{\textbf{u}: \text{val}(c_{\textbf{u}}) + \text{val}(\textbf{y}^{\textbf{u}'}) + w_{n}\cdot u_{n} = \text{trop}(g)(w_{n})} \overline{t^{-\text{val}(c_{\textbf{u}})}c_{\textbf{u}}t^{-\textbf{u}'\cdot \textbf{w}'}y^{\textbf{u}'}}x^{u_n}_{n} \\
            =& \sum_{\textbf{u}: \text{val}(c_{\textbf{u}}) + \textbf{w} \cdot \textbf{u} = \text{trop}(g)(w_{n})}\overline{t_{-\text{val}(c_{\textbf{u}})}c_{\textbf{u}}}\alpha^{\textbf{u}'}x^{u_{n}}_{n}\\
            =& ~ \text{in}_{\textbf{w}}(f)(\alpha_{1}, \dots, \alpha_{n-1}, x_{n}).
        \end{align*}
        So from $\text{in}_{\textbf{w}}(f)(\alpha) = 0$ we get $\text{in}_{w_{n}}(g)(\alpha_{n}) = 0$.
        From the $n=1$ case there exists a $y_{n} \in K^{*}$ such that $\text{val}(y_n) = w_n$, $\overline{t^{-\text{val}(w_{n})}y_{n}} = \alpha_{n}$, and $g(y_{n}) = 0$, implying $f(y_1,\dots,y_n) = 0$.
        Therefore we have found the required point $y \in V(f)$.
 %       \par \textcolor{red}{Finish last statement of prrof}

    \end{proof}

\subsection{Structure Theorem?}
    \begin{theorem}
        For $f \in K[x_1^{\pm 1}, \dots, x_n^{\pm 1}]$, and $\text{trop}(V(f)) = |\Delta|$. Where $\Delta$ is a pure $\Gamma_{val}^n-$rational polyhedral complex of dimension $n-1$, more specifically:
        \begin{equation*}
            \Delta = \left(\text{dual}\left(\text{r-subdiv}_{\text{val}(c_u)}\left(\text{Newt}(f)\right)\right)\right)^{[n-1]}
        \end{equation*}
    \end{theorem}
    \begin{proof} 
       Let $f = \sum_{u \in \mathbb{Z}^n}c_ux^u$
       \begin{itemize}
           \item $P = \text{conv}\{u: c_u \neq 0\} \subset \mathbb{R}^n$ .
           \item $P_{\text{val}} = \text{conv}\{(u,c_u): c_u \neq 0\}\subset \mathbb{R}^{n+1}$.
       \end{itemize}
       Defining the following will be helpful:
       \begin{itemize}
           \item $\text{face}_v(P) = \{x \in P : x\cdot v \leq y \cdot v~ \text{for all } y \in P\}$. Here the vector $v$ is an inward pointing normal vector of the face.
           \item $\pi: \mathbb{R}^{n+1} \to \mathbb{R}^n$ is projection onto first $n$-coords.
       \end{itemize}
       The regular subdivision of $P$ with weight vectors given by $c_u$ is the projection of the lower faces of $P_{\text{val}}$.
       A face is called a \textit{lower face} if it can be defined with a vector $v$ for which $v_{n+1} > 0$.
       
       Now that we have the regular subdivision, we define the dual:
       \begin{itemize}
           \item The \textit{Normal Cone} of a face is the set of all defining vectors of the face.
               Formally, for a face $F \subset P_{\text{val}}$, the \textit{Normal Cone} $\mathcal{N}(F) = \{v \in \mathbb{R}^{n+1}: \text{face}_v(P_{\text{val}}) = F\}$.
           \item And $\mathcal{F} = \{v \in \mathcal{N}(F): v_{n+1} = 1\}$.
           \item $\tilde{\pi}(F) = \pi(\mathcal{F})$
       \end{itemize}
       And the complex $D = \{\tilde{\pi}(F) : F \text{ is a lower face of }P_{\text{val}}\}$ defines the dual of the regular subdivision. 
       Let's see what the initial form defined by $v \in \tilde{\pi}(F)$ looks like (let $\bar{v} = (v,1) \in \mathcal{F}$ be the pre-image of $v$).
       \begin{align*}
           \text{in}_v(f) = \sum_{\text{val}(c_u) + v \cdot u = w} \overline{c_u t^{-\text{val}(c_u)}} x^u
       \end{align*}
       Where $w = \text{min}_{u}\{\text{val}(c_u) + u\cdot v\}$. 
       Notice $\text{val}(c_u) + v \cdot u = \bar v \cdot (u,c_u)$, 
       so if $x^{u'}$ has non-zero coefficient in $\text{in}_v(f)$: 
       \begin{equation*}
           (u',c_{u'})\cdot \bar{v} \leq (u,c_u) \cdot \bar{v}~~~\forall u
       \end{equation*}
       Which is precisely the condition required for $(u,c_u)$ to be a vertex of the face F defined by $\bar v$. 
       And consequently for $u$ to be vertex of $\pi(F)$. 
       Therefore for $\text{in}_{v}(f)$ is \textit{not} a monomial \textbf{iff} $\text{face}_v(P_{val})$ has more than vertex.
       \par So $w\in \text{trop}(f)(v)$ \textbf{iff} $\text{face}_{(w,1)}(P_{val})$ is a \textit{not} a vertex. 
       Implying, $w \in D^{[n-1]}$.
   \end{proof}
\subsection{Combinatorial Description of Tropical Curves}

\section{Gr\"{o}bner and Tropical Bases}  
    \begin{lemma}
        $I \subset K[x_0,\dots, x_n]$ be a homogenous ideal; fix $w\in \mathbb{R}^{n+1}$. 
        Then $\text{in}_{w}(I)$ is a homogenous ideal, and there exists a homogenous greobner basis for $I$. 
        Further, if $w \in \Gamma_{\text{val}}$ then $g \in \text{in}_{w}(I)$ has the form $g = \text{in}_{w}(f)$ for $f \in I$.
    \end{lemma}

    \begin{lemma}
        For a fixed polynomial $f \in K[x_0,\dots,x_n]$ and $w,v \in \mathbb{R}^{n+1}$ there exists $\varepsilon >0$ such that for all $0<\varepsilon ' < \varepsilon$ we have:
        \begin{equation*}
            \text{in}_{v}(\text{in}_{w}(f)) = \text{in}_{w + \varepsilon 'v}(f)
        \end{equation*}
    \end{lemma}
    
    \begin{lemma}
        If $I \subset K[x_0,\dots,x_n]$ is a homogenous prime ideal of dimension $d$ and $w \in \Gamma_{\text{val}}^{n+1}$, then every minimal associated prime of $\text{in}_{w}(I)$ has dimension $d$.
    \end{lemma}

\subsection{Tropical Bases}

A tropical basis $\mathcal{T}\subset K[x_1^{\pm 1}, \dots, x_n^{\pm 1}]$ is a \textit{finite generating set} such that, if for all $w \in \mathbb{R}^{n}$:
    \begin{align*}
        \exists ~f \in I~\text{such that }\text{trop}(f)(w)&~\text{achieves minimum only once}\\
                                                          & \textbf{iff}\\
        \exists ~g \in \mathcal{T}~\text{such that }\text{trop}(g)(w)&~\text{achieves minimum only once}.
    \end{align*}

    Before stating the fundamental and structure theorem for arbritary tropical varieties we define the notion of a tropical morphism.
    Remember that a morphism of algebraic groups $\phi : T^{n}\to T^{m}$ induces a monomial map $\phi^{*}:K[x_1^{\pm 1}, \dots, x_m^{\pm 1}]\to K[z_1^{\pm 1}, \dots, z_n^{\pm 1}]$. 
    We can also represent this monomial map as a map $\phi^{*}: \mathbb{Z}^{m}\to \mathbb{Z}^{n}$ where $\phi^{*}(e_i) = u$ for $\phi^{*}(x_i) = z^u$. We now define the \textbf{tropicalization of $\phi$} as the map
    \begin{equation*}
        \text{trop}(\phi): \text{Hom}(\mathbb{Z}^{n},\mathbb{Z})\cong \mathbb{Z}^{n} \to \text{Hom}(\mathbb{Z}^{m},\mathbb{Z})\cong \mathbb{Z}^{m}
    \end{equation*}

\subsection{Gr\"{o}bner Complexes}

\section{Arbritary Tropical varieties}
\subsection{Fundamental Theorem}

    \begin{theorem}[Fundamental Theorem of Tropical Algebraic Geometry]
        Let $K$ be algebraically closed with a non trivial valuation, and let $I \subset K[x_1^{\pm 1}, \dots, x_n^{\pm 1}]$ with $X = V(I)$ a very affine variety. 
        Then the following three sets are the same:
        \begin{enumerate}
            \item $\text{trop}(X)$.
            \item $\{w \in \mathbb{R}^{n}: \text{in}_{w}(I) \neq \langle 1\rangle\}$. 
            \item the closure of set $\left\{(\text{val}(y_1), \dots, \text{val}(y_n)): (y_1,\dots, y_n) \in X\right\}$.
        \end{enumerate}
        Further, if $X$ is irreducible and $w \in \Gamma_{\text{val}} \cap \text{trop}(X)$, then the set $\{y \in X: \text{val}(y) = w\}$ is Zariski dense in $X$.
    \end{theorem}
\subsection{Structure Theorem}
    \begin{theorem}[Structure Theorem for Tropical Varieties]
        Let $X$ be an \textit{irreducible d-diminsional} subvariety of $T^{n}$. Then, trop$(X)$ is the support of a balanced-weighted $\Gamma_{\text{val}}-$rational purely d-dimensional polyhedral complex, which is connected through codimension $1$.
    \end{theorem}
\subsection{Stable Intersection}

\section{Berkovich Analytic Spaces}
\subsection{The Berkovich Analytification}
    In this section the norm $|\cdot |:K \to \mathbb{R}$ is defined as $|x| = \text{exp}(-\text{val}(x))$. This norm is \textit{non-archimedian}, that is: 
mm
    \begin{itemize}
        \item $|a| = 0$ \textbf{iff} $a=0$
        \item $|ab| = |a||b|$
        \item $|a+b| \leq |a|+|b|$
    \end{itemize}
    We also assume $K$ is complete in this section.
    \begin{definition}
        Consider and affine scheme, $X = \text{Spec}(A)$, of finite type over $K$. 
        The \textbf{Berkovich analytic space} $X^{an}$ is the set of seminorms on $A$ which extend the norm $|\cdot|$ on $K$. 
        So $p:A \to \mathbb{R}_{+}$ is a seminorm if:
        \begin{enumerate}
            \item $p(fg) = p(f)p(g)$
            \item $p(1)=1$
            \item $p(f+g)\leq p(f)+p(g)$
            \item $p(\alpha) = |\alpha|$ when $\alpha \in K$
        \end{enumerate}
        This set is given the coarsest topology such that the maps $\phi_f:X^{an}\to \mathbb{R}_{+}$ given by $\phi_f(p) = p(f)$ are continuous for all $f \in A$.
    \end{definition}
    For a point $p \in X^{an}$ we can construct a valued field $(L,w)$ over $K$, along with the $L-$rational point $P$ in $\text{Spec}(A)$ such that for all $a \in A$, $p(a) = |a(P)|_{w}$. 
    Here $L = A/\{a \in A: p(a)=0\}$. 
    Conversely given a valued extension $(L,w)$ of $K$ and any $L-$rational point $P$ of $A$ gives an element $p \in X^{an}$ defined by $p(a) = |a(P)|_{w}$.
\subsection{Tropicalisation}

    Consider a free abelian group $M \cong \mathbb{Z}^n$ of rank n, and define $N = \text{Hom}(M,\mathbb{Z})$. 
    Then $T:= \text{Spec}(K[M])$ is the multiplicative torus with character group $M$. 
    Let $N_{\mathbb{R}} = \mathbb{R} \otimes N = \text{Hom}(M,\mathbb{R})$. Then the \textbf{tropicalization map} is defined as:
    \begin{align*}
        \text{trop}_v:T^{an} &\to N_{\mathbb{R}}\\
        p &\mapsto \text{trop}_v(p)
    \end{align*}
    Such that for the character $\chi^u$ of $T$ corresponding to $u \in M$, $\langle u,\text{trop}_{v}(p)\rangle = -\text{log}(p(\chi^{u}))$. So if choose some coodinates $x_1, \to, x_n$ for $T^{n}$ then, $\text{trop}_v(p) = (-\text{log}(p(x_1)), \dots,\allowbreak -\text{log}(p(x_n)))$.
    \par So for any affine scheme of finite type $X$, we define the \textbf{tropical variety associated to }$X$ as $\text{Trop}_v(X) = \text{trop}_v(X^{an})$. 
    This construction makes many tropical objects easier to deal with. 
    For example if $X$ is connected then we know $X^{an}$ is connected (see \cite{berkovich2012spectral}), since $\text{trop}$ is a continuous map it's image also must be connected, proving that the tropical variety is also connected.
\subsection{Another Proof of the Fundamental Theorem}
