\section{Introduction}
In 2005 Grigory Mikhailkin showed that the number of degree $d$ plane tropical curves passing through $3d-1$ points, $N^{\text{trop}}_{d}$ is same as the number of plane rational curves $N_{d}$.
This result brought tropical geometry into the eyes of the wider mathematical community, and also gave rise to a new array of research problems.
In this chapter we are going to look at Gathmann and Markwig's work which proved Kontsevich's curve counting formula for tropical curves without making use of Mikhailkin's correspondence result.

\subsection{Overview of Proof Idea}
Before we show how to proceed with this count of tropical curves, it's instructive to look at the proof from the previous chapter for general ideas.
The proof in for rational plane curves in chapter 2 followed the following steps:
\begin{enumerate}
    \item For degree $d$ curves, we considered the moduli space $\overline{\mathcal{M}}_{0,3d}(\mathbb{P}^{2},d)$.
    \item The subvariety $Y \subset \overline{\mathcal{M}}_{0,d}(\mathbb{P}^{2},d)$ defined by:
        \[
            Y:= \nu_{1}^{-1}(L_{1}) \cap \nu_{2}^{-1}(L_{2}) \cap \nu_{3}^{-1}(Q_{1}) \cap \dots \cap \nu_{3d}^{-1}(Q_{3d-2}),
        \]
        corresponds to the set of degree $d$ curves passing through the lines $L_{1}$ and $L_{2}$ and points $Q_{i}$.
    \item For the forgetful map, $\eta: \overline{\mathcal{M}}_{0,3d}(\mathbb{P}^{2},d)\to \overline{\mathcal{M}}_{0,4}$, look at the inverse image divisors of $D(m_{1}, m_{2}|p_{1}, p_{2})$ and $D(m_{1}, p_{1}|p_{2}, m_{2})$.
        Since these divisors are linearly equivalent we get the relation, 
        \[
            Y \cap D(m_{1}, m_{2} | p_{1}, p_{2}) \equiv Y \cap D(m_{1}, p_{1} | m_{2}, p_{2}).
        \]
    \item Equating the degrees of the divisors on the two sides of the above relation we obtain Kontsevich's formula.
\end{enumerate}

Ideally we would like to repeat these same steps in the tropical setting to obtain the curve counting formula, but we lack the tools to do so.
The moduli spaces of tropical curves aren't as well behaved as moduli spaces of rational plane curves.
In fact, the moduli space of tropical curves isn't even a tropical variety.
Further, we don't have a rich theory of divisors to rely upon for tropical varieties. 
But we do have enough tools in the tropical world to sketch out a proof of the curve counting formula in the spirit of Chapter 2.
\begin{enumerate}
    \item kk
\end{enumerate}

\section{The Prerequisites}
The enumerative problem we are going to tackle just considers tropical curves in $\mathbb{R}^{2}$.
As we saw in \textcolor{red}{chapter $1$}, these tropical curves just look like plane graphs.
We use this observation to give an equivalent definiton of tropical curves in $\mathbb{R}^{2}$.
This description of tropical curves will also make it easier for us to draw motivation from ideas discussed in the previous chapter, like the space of stable maps.

\par In the tropical world, the moduli spaces aren't tropical varieties themselves. 
This is in contrast with the picture of rational plane curves.
A consequence of this disconnect between the two worlds is that we can't exploit the properties of divisors like we did earlier.
We have to be careful and consider ideas which help us make similar manipulations with tropical moduli spaces.

\subsection{Tropical curves and their Moduli}

\begin{definition}[Graphs]
    A graph $G$ is a pair $(V,E)$ of sets of vertices and edges.
    Where each element of the set of edges $E$ is a tuple of vertices $\{v_{i},v_{j}\}$ for $v_{i} \neq v_{j}$ in $V$. 
    For every edge $\{v_{i},v_{j}\} \in E$, there exists multiple orderings $(v_{i},v_{j})$ and $(v_{j},v_{i})$. We refer to each such ordering of the edge as a flag.
    The valence of a vertex $v$, $\nu(v)$, is defined as the number of edges incident to $v$, that is: $\nu(v):=|\{e \in E: v\in e\}|$.
\end{definition}

\begin{definition}[Graph without tips]
    For a graph $G = (V,E)$ the, if we define $\mathcal{V}_{1} = \{v \in V: \nu(v) =1  \}$, then for $V':= V\backslash \mathcal{V}_{i}$ the tuple $\tilde{G} = (V',E)$ will be referred to as a graph without tips. 
    Every edge incident to vertices in the set $\mathcal{V}_{1}$ will be referred to as an end of the graph $\tilde G$.
    Further, for each end of the graph there exists only one corresponding flag.
\end{definition}

\begin{remark}
    From now on, when we refer to graphs, we mean a graph without tips.
\end{remark}

\begin{definition}[Metric Graph]
    Consider a pair $(G,l)$, where $l$ is a function $E \to \mathbb{R}_{+} \cup \infty$, and $G$ a graph (without tips).
    Here, $l(e) = \infty$ \textbf{iff} $e$ is an end of the graph $G$.
For each edge $e$ consider the interval $I_{e} = [0,l(e)]\subset \mathbb{R}$ (or $[0,\infty)$).
\end{definition}

\begin{definition}[Abstract Tropical Curves]
    aa
\end{definition}

\begin{definition}[Moduli of Abstract Tropical curves]
    
\end{definition}

\begin{definition}[Plane Tropical Curves]
    
\end{definition}

\begin{definition}[Moduli of Tropical plane curves]
    
\end{definition}

\begin{proposition}[Equivalence between the Two Definitions]
    
\end{proposition}

\begin{definition}[Combinatorial Types]
    
\end{definition}

\begin{lemma}[Finite number of Combinatorial types]
    
\end{lemma}

\begin{lemma}
    Every $3$-valent tropical curve with $n$ unbounded edges has $n-3$ bounded edges.
\end{lemma}

\begin{remark}[Tropical Moduli spaces are polyhedral complexes]
    
\end{remark}

\subsection{Multilplicites and Evaluation Maps}

\begin{definition}[Morphism between polyhedra]
    
\end{definition}

\begin{definition}[Multiplicity of polyhedral morphisms]
    
\end{definition}

\begin{definition}[Degree of a Polyhedral Morphism]
    
\end{definition}

\begin{definition}[Evaluation Maps]
    
\end{definition}

\begin{remark}[Degree of total evaluation counts number of curves]
    
\end{remark}

\begin{definition}[Rigid plane tropical curves]
    
\end{definition}

\begin{definition}[Multiplicity of a Plane tropical curve]
    
\end{definition}
\subsection{Forgetful Maps}
\begin{definition}[Forgetful map definition]
    
\end{definition}

\begin{definition}[The total map we will use to define Kontsebich's formula]
    
\end{definition}

Some properties of the map $\pi$ And what can it count?
\section{Kontsevich's Formula}
